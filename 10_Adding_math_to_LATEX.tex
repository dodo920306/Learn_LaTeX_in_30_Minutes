\documentclass{article}
\usepackage{amsmath} % For the equation* environment
\nofiles
\begin{document}

% To typeset inline-mode math you can use one of these delimiter pairs: \( ... \), $ ... $ or \begin{math} ... \end{math}.
In physics, the mass-energy equivalence is stated 
by the equation $E=mc^2$, discovered in 1905 by Albert Einstein.

\begin{math}
    E=mc^2
\end{math} is typeset in a paragraph using inline math mode---as is $E=mc^2$, and so too is \(E=mc^2\).

% To typeset display-mode math you can use one of these delimiter pairs:
% \[ ... \], \begin{displaymath} ... \end{displaymath} or \begin{equation} ... \end{equation}.
The mass-energy equivalence is described by the famous equation
\[ E=mc^2 \] discovered in 1905 by Albert Einstein. 

In natural units ($c = 1$), the formula expresses the identity
% By using \begin{equation} ... \end{equation}, the content will be numbered.
\begin{equation}
E=m
\end{equation}

Subscripts in math mode are written as $a_b$ and superscripts are written as $a^b$.
These can be combined and nested to write expressions such as

\[ T^{i_1 i_2 \dots i_p}_{j_1 j_2 \dots j_q} = T(x^{i_1},\dots,x^{i_p},e_{j_1},\dots,e_{j_q}) \]

We write integrals using $\int$ and fractions using $\frac{a}{b}$.
Limits are placed on integrals using superscripts and subscripts:

\[ \int_0^1 \frac{dx}{e^x} =  \frac{e-1}{e} \]

Lower case Greek letters are written as $\omega$ $\delta$ etc.
while upper case Greek letters are written as $\Omega$ $\Delta$.

Mathematical operators are prefixed with a backslash as $\sin(\beta)$, $\cos(\alpha)$, $\log(x)$ etc.

\section{First example}

The well-known Pythagorean theorem \(x^2 + y^2 = z^2\) was proved to be invalid for other exponents, 
meaning the next equation has no integer solutions for \(n>2\):

\[ x^n + y^n = z^n \]

\section{Second example}

This is a simple math expression \(\sqrt{x^2+1}\) inside text. 
And this is also the same: 
\begin{math}
    \sqrt{x^2+1}
\end{math}
but by using another command.

This is a simple math expression without numbering
\[\sqrt{x^2+1}\] 
separated from text.

This is also the same:
\begin{displaymath}
    \sqrt{x^2+1}
\end{displaymath}

% \ldots is used in both text and math mode for ellipsis in sentences and mathematical expressions, 
% while \dots is specifically designed for math mode and is used for ellipsis in sequences, series, and other mathematical contexts.
\ldots and this:
\begin{equation*}
    \sqrt{x^2+1}
\end{equation*}

\end{document}
