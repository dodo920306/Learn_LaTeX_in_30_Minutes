\documentclass{article}
% LaTeX package to import graphics
\usepackage{graphicx}
% configuring directory path of the graphicx package
\graphicspath{{images/}}
\begin{document}
The universe is immense and it seems to be homogeneous, on a large scale, everywhere we look.

% The \includegraphcs command is provided (implemented) by the graphicx package
% it’s considered best practice to omit the file extension
% because it will prompt LATEX to search for all the supported formats.
% Generally, the graphic’s file name should not contain white spaces or multiple dots;
% it is also recommended to use lowercase letters for the file extension when uploading image files to Overleaf.
\includegraphics{universe}  
 
There's a picture of a galaxy above.

% .aux file is needed for image labelling.
% Images can be captioned, labelled and referenced by means of the figure environment.
% h stands for "here," 
% indicating that LaTeX should try to place the figure approximately at the location in the text where it is defined.
\begin{figure}[h]
    \centering
    \includegraphics[width=0.75\textwidth]{mesh}
    \caption{A nice plot.}
    \label{fig:mesh1}
\end{figure}

% Twice compilation is needed for getting the .aux file and correcting references respectively.
As you can see in figure \ref{fig:mesh1}, the function grows near the origin. This example is on page \pageref{fig:mesh1}.

% t : Places the float at the top of the page.
\begin{figure}[t]
    \centering
    \includegraphics[width=0.75\textwidth]{mesh}
    \caption{A nice plot.}
\end{figure}

% b : Places the float at the bottom of the page.
\begin{figure}[b]
    \centering
    \includegraphics[width=0.75\textwidth]{mesh}
    \caption{A nice plot.}
\end{figure}

% p : Places the float on a separate page containing only floats.
\begin{figure}[p]
    \centering
    \includegraphics[width=0.75\textwidth]{mesh}
    \caption{A nice plot.}
\end{figure}

% If LaTeX cannot find a suitable place for the figure based on the document layout and other content,
% it will make its own decision on where to place it.

\end{document}
